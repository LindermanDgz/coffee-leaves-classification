\chapter{Introducción}

\section{Procesamiento Digital de Imágenes}
Una imagen digital puede considerarse como una función 2-dimensional que representa puntos en el espacio llamados pixeles los cuales son finitos y discretos, y que presentan una intensidad definida. El \textit{Procesamiento Digital de Imágenes} hace referencia al proceso de manipular imágenes digitales mediante el uso de computadoras.

La visión es nuestro sentido más avanzado, sin embargo, estamos limitados a las capacidades biológicas de nuestros ojos para la percepción de imágenes. Una computadora en cambio tiene la facilidad de poder adaptar su espectro visual más allá de lo que nosotros somos capaces de ver.

No existe una frontera bien definida sobre dónde el procesamiento de imágenes termina respecto a otras áreas como el análisis de imágenes y la visión por computadora. Sin embargo, podemos considerar tres tipos de procesos computarizados que determinan el nivel de procesamiento llevado acabo.

Los procesos de bajo nivel que involucran operaciones primitivas tales como la reducción de ruido o la mejora del contraste. Estos procesos reciben como entrada una imagen y producen como salida otra imagen. Los procesos de nivel intermedio, que usualmente reciben una imagen de entrada pero que producen como salida los atributos o características de la imagen proporcionada, tales como bordes, contornos, etc. Y finalmente los procesos de alto nivel que buscan darle sentido a las imágenes procesadas aplicando funciones congnitivas generalmente asociadas a la visión.

En general, el \textit{Procesamiento Digital de Imágenes} involucra los procesos que reciben y generan imágenes y que opcionalmente pueden extraer sus características. \cite{gonzalez2008digital}

\section{Procesamiento Digital de Imágenes: Fundamentos y Aplicaciones con GNU Octave y OpenCV}
Durante los meses de junio a agosto de 2025 se impartió el seminario \textit{Procesamiento Digital de Imágenes: Fundamentos y Aplicaciones con GNU Octave y OpenCV} a través de la \textit{Benemérita Universidad Autónoma de Chiapas} y dicatado por el \textit{PhD. Luis Escalante Zárate} en el cual se revisaron los conceptos y métodos tradicionales del procesamiento digital de imágenes tales como filtros, segmentación, creación de máscaras, histogramas, espacios de color, transformaciones en el dominio de la frecuencia, etc. de manera práctica e intuitiva en sesiones diarias de seis horas por un total de seis semanas y dos más de revisiones personalizadas.

Las herramientas utilizadas en el curso fueron muy variadas como \textit{IrfanView} para la visualización y edición rápida de imágenes, \textit{Python} con \textit{OpenCV} para prueba de conceptos de manera agilizada, y destacando sobre todos el uso de \textit{GNU Octave} como una alternativa \textit{open-source} a la herramienta de \textit{MatLab}.

A partir de este seminario surge la necesidad del desarrollo de un proyecto que exponga el conocimiento adquirido y que sirva como memoria del curso.