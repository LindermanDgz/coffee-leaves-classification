\chapter{Planteamiento del problema}

\section{Clasificación de hojas de café}

\section{Objetivo general}
Demostrar las habilidades adquiridas durante el seminario \textit{Procesamiento Digital de Imágenes: Fundamentos y Aplicaciones con GNU Octave y Open CV} aplicando los principios y técnicas básicas de manera práctica a un proyecto en particular.

\section{Objetivo específico}
Crear un algoritmo que clasifique hojas de café como sanas o infectadas y su nivel de afectación, y evaluar su eficiencia comparando los resultados obtenidos con los proporcionados en el conjunto de datos.

\section{Alcance}
A pesar de que el conjunto de datos de prueba contiene seis clasificaciones para las hojas, el algoritmo desarrollado sólo incluirá la clasificación sana y los cuatro niveles de afectación, excluyendo la clasificación \textit{araña roja} debido a las retricciones en el tiempo del proyecto.

\section{Justificación}
El algoritmo y las técnicas utilizadas pueden aplicarse de manera directa en el mundo real dentro del área de la agricultura y/o agronomía, e idealmente puede servir como base para desarollar procesos automatizados para el control de calidad en el ámbito del café y control de plagas.
 
\section{Especificaciones técnicas}
Se utilizará Python como lenguaje de programación para la implementación del algoritmo debido a su facilidad de uso y al amplio número de bibliotecas disponibles para el procesamiento de imágenes tales como OpenCV.

\section{Metodología}

\subsection{Modelo de colores HSV}