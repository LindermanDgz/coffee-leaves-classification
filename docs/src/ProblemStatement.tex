\chapter{Planteamiento del problema}
El café es un producto altamente consumido en el mundo y es una de las bebidas más populares y preferidas en la sociedad. La producción de café representa un sustento económico para muchas familias en México y más precisamente en Chiapas que es uno de los principales estados exportadores de este grano.

El precio del café tuvo un incremento en el último año, sin embargo, las plagas representan un riesgo para su producción. Algunas plagas como la roya, la araña roja o la broca, pueden causar perdidas severas tanto económicas como de materia prima que bien podrían minimizarse o incluso prevenirse mediante un monitoreo adecuado y regular de la plantación evitando medidas drásticas correctivas como la aplicación de químicos.

El monitoreo de los cultivos de café generalmente require de capacitación o asesoría técnica especializada que muchas veces es cara o inaccesible. El proyecto que aquí se presenta busca ofrecer una alternativa para quienes apenas ingresan en el mundo del café o bien si no se cuenta con los recursos de inversión suficientes.

\section{Clasificación de hojas de café infectadas por roya usando una segmentación por matices en el canal Hue}

\section{Objetivo general}
Demostrar las habilidades adquiridas durante el seminario \textit{Procesamiento Digital de Imágenes: Fundamentos y Aplicaciones con GNU Octave y Open CV} aplicando los principios y técnicas básicas de manera práctica a un proyecto en particular.

\section{Objetivo específico}
Crear un algoritmo que clasifique hojas de café como sanas o infectadas por roya y su nivel de afectación, y evaluar su eficiencia comparando los resultados obtenidos con los proporcionados en el conjunto de datos.

\section{Alcance}
A pesar de que el conjunto de datos de prueba contiene seis clasificaciones para las hojas, el algoritmo desarrollado sólo incluirá la clasificación sana y los cuatro niveles de afectación por roya, excluyendo la clasificación \textit{araña roja} debido a las retricciones en el tiempo del proyecto.

\section{Justificación}
El algoritmo y las técnicas utilizadas pueden aplicarse de manera directa en el mundo real dentro del área de la agricultura y/o agronomía, e idealmente puede servir como base para desarollar procesos automatizados para el control de calidad en el ámbito del café y control de plagas.
 
\section{Especificaciones técnicas}
Se utilizará \textit{Python} como lenguaje de programación para la implementación del algoritmo debido a su facilidad de uso y al amplio número de bibliotecas disponibles para el procesamiento de imágenes tales como \textit{OpenCV}. Así mismo, se utilizará una libreta de \textit{Jupyter} para la ejecución del algoritmo debido a su flexibilidad para ejecutar trozos de código de manera interactiva.

\section{Metodología}

\subsection{Modelo de colores HSV}