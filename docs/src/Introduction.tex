\chapter{Introducción}

\section{Procesamiento Digital de Imágenes}

\section{Procesamiento Digital de Imágenes: Fundamentos y Aplicaciones con GNU Octave y OpenCV}
Durante los meses de junio a agosto de 2025 se impartió el seminario \textit{Procesamiento Digital de Imágenes: Fundamentos y Aplicaciones con GNU Octave y OpenCV} a través de la \textit{Benemérita Universidad Autónoma de Chiapas} y dicatado por el \textit{PhD. Luis Escalante Zárate} en el cual se revisaron los conceptos y métodos tradicionales del procesamiento digital de imágenes tales como filtros, segmentación, creación de máscaras, histogramas, espacios de color, transformaciones en el dominio de la frecuencia, etc. de manera práctica e intuitiva en sesiones diarias de seis horas por un total de seis semanas y dos más de revisiones personalizadas.

Las herramientas utilizadas en el curso fueron muy variadas como \textit{IrfanView} para la visualización y edición rápida de imágenes, \textit{Python} con \textit{OpenCV} para prueba de conceptos de manera agilizada, y destacando sobre todos el uso de \textit{GNU Octave} como una alternativa \textit{open-source} a la herramienta de \textit{MatLab}.

A partir de este seminario surge la necesidad del desarrollo de un proyecto que exponga el conocimiento adquirido y que sirva como memoria del curso.